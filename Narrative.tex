% Options for packages loaded elsewhere
\PassOptionsToPackage{unicode}{hyperref}
\PassOptionsToPackage{hyphens}{url}
%
\documentclass[
]{article}
\usepackage{amsmath,amssymb}
\usepackage{iftex}
\ifPDFTeX
  \usepackage[T1]{fontenc}
  \usepackage[utf8]{inputenc}
  \usepackage{textcomp} % provide euro and other symbols
\else % if luatex or xetex
  \usepackage{unicode-math} % this also loads fontspec
  \defaultfontfeatures{Scale=MatchLowercase}
  \defaultfontfeatures[\rmfamily]{Ligatures=TeX,Scale=1}
\fi
\usepackage{lmodern}
\ifPDFTeX\else
  % xetex/luatex font selection
\fi
% Use upquote if available, for straight quotes in verbatim environments
\IfFileExists{upquote.sty}{\usepackage{upquote}}{}
\IfFileExists{microtype.sty}{% use microtype if available
  \usepackage[]{microtype}
  \UseMicrotypeSet[protrusion]{basicmath} % disable protrusion for tt fonts
}{}
\makeatletter
\@ifundefined{KOMAClassName}{% if non-KOMA class
  \IfFileExists{parskip.sty}{%
    \usepackage{parskip}
  }{% else
    \setlength{\parindent}{0pt}
    \setlength{\parskip}{6pt plus 2pt minus 1pt}}
}{% if KOMA class
  \KOMAoptions{parskip=half}}
\makeatother
\usepackage{xcolor}
\usepackage[margin=1in]{geometry}
\usepackage{graphicx}
\makeatletter
\def\maxwidth{\ifdim\Gin@nat@width>\linewidth\linewidth\else\Gin@nat@width\fi}
\def\maxheight{\ifdim\Gin@nat@height>\textheight\textheight\else\Gin@nat@height\fi}
\makeatother
% Scale images if necessary, so that they will not overflow the page
% margins by default, and it is still possible to overwrite the defaults
% using explicit options in \includegraphics[width, height, ...]{}
\setkeys{Gin}{width=\maxwidth,height=\maxheight,keepaspectratio}
% Set default figure placement to htbp
\makeatletter
\def\fps@figure{htbp}
\makeatother
\setlength{\emergencystretch}{3em} % prevent overfull lines
\providecommand{\tightlist}{%
  \setlength{\itemsep}{0pt}\setlength{\parskip}{0pt}}
\setcounter{secnumdepth}{-\maxdimen} % remove section numbering
\ifLuaTeX
  \usepackage{selnolig}  % disable illegal ligatures
\fi
\usepackage{bookmark}
\IfFileExists{xurl.sty}{\usepackage{xurl}}{} % add URL line breaks if available
\urlstyle{same}
\hypersetup{
  pdftitle={Narrative},
  pdfauthor={Madeline Morman},
  hidelinks,
  pdfcreator={LaTeX via pandoc}}

\title{Narrative}
\author{Madeline Morman}
\date{2/28/2025}

\begin{document}
\maketitle

Cherry blossoms are not only a breathtaking natural phenomenon but also
serve as a critical biological indicator of climate patterns and
environmental change. The goal of this project is to build a prediction
model that will accurately predict when the beautiful cherry blossoms
will bloom around the world at 5 sites in the year 2025. Numerous
predictors, including location, year, and weather, are included in the
dataset used in my prediction model. The average weather data was split
into chilling and warming periods in each location because these periods
are crucial in the development of the flower and when it will bloom.
These bloom dates are essential for agricultural planning and
comprehending how climate change affects seasonal cycles, thus we are
working to create a reliable model that can predict them.

To accomplish this goal I used a lasso regression method which is a
regularized linear regression technique used to handle high-dimensional
data and avoid over fitting. This model used the data from 1973 to 2023
to train on and then was tested using the data from beyond 2023 and then
evaluated using 5-fold cross-validation. This project is one of many
examples in how statistical modeling can be applied to to further our
understanding and predictive capabilities of the timing of certain
seasonal events. While this model was created to predict cherry
blossoms, similar ones can be made that could help offer insights into
the future effects of climate change.

\end{document}
